\markright{correlmx}
\hspace*{-1.6cm}{\Large \bf parafrep}

\vspace*{-.4cm}
\hspace*{-1.6cm}\rule[0in]{16.5cm}{.02cm}
\vspace*{.2cm}



{\bf \large \fontfamily{cmss}\selectfont Purpose}\\
\hspace*{1.5cm}
\begin{minipage}[t]{13.5cm}
computes a parametric frequency representation of a signal.
\end{minipage}
\vspace*{.5cm}


{\bf \large \fontfamily{cmss}\selectfont Synopsis}\\
\hspace*{1.5cm}
\begin{minipage}[t]{13.5cm}
\begin{verbatim}
[spec,freqs,A]=parafrep(Rx,N,method)
[spec,freqs,A]=parafrep(Rx,N,'genlag',q)
\end{verbatim}
\end{minipage}
\vspace*{.5cm}


{\bf \large \fontfamily{cmss}\selectfont Description}\\
\hspace*{1.5cm}
\begin{minipage}[t]{13.5cm}
{\ty correlmx} computes a correlation matrix of a signal.\\

\hspace*{-.5cm}
\begin{tabular*}{14cm}{p{1.5cm} p{8.5cm} c}
Name & Description & Default value\\\hline
{\ty x}      & analyzed signal &  \\ 
{\ty p}      & highest autocorrelation lag. For a given $p$, 
               the size of the autocorrelation matrix will be 
               $(p+1)\times(p+1)$ & \\
{\ty Rxtype} & computation algorithm (default : 'fbhermitian')   & {\ty 'fbhermitian'}\\\hline 
{\ty Rx}     & computed correlation matrix, of size \hbox{$(p+1)\times(p+1)$} & \\\hline 
\end{tabular*}

\end{minipage}
\vspace*{1cm}


{\bf \large \fontfamily{cmss}\selectfont Example}\\
\hspace*{1.5cm}
\begin{minipage}[t]{13.5cm}
\begin{verbatim}
N=100; sig=real(fmconst(N,0.1))+0.4*randn(N,1); 
Rx=correlmx(sig,2,'burg'); [v,d] = eig(Rx), acos(-0.5*v(2,1)/v(1,1))/(2*pi)
Rx=correlmx(sig,2,'hermitian'); [v,d] = eig(Rx), acos(-0.5*v(2,1)/v(1,1))/(2*pi)
\end{verbatim}
computes an estimation of the frequency of a sinusoid using the Pisarenko
approach, with a correlation matrix computed by the Burg or the hermitian
algorithm.

\end{minipage}
\vspace*{.5cm}


{\bf \large \fontfamily{cmss}\selectfont See Also}\\
\hspace*{1.5cm}
\begin{minipage}[t]{13.5cm}
\begin{verbatim}
parafrep, tfrparam
\end{verbatim}
\end{minipage}
