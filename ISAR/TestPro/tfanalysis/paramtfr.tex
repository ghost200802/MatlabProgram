\documentclass[12pt,a4paper]{report}
\usepackage{times}

\newcommand{\ty}{\ttfamily}
\newcommand{\esp}[1]{I\!\!E\left[{#1}\right]}
\newcommand{\un}{1\!\!{\rm I}}

%%%%%%%%%%%%%%%%%%%%%%% headline routines %%%%%%%%%%%%%%%%%%%%%%%%%%%% 
%
%  \pagestyle{STYLE}     : sets the page style of the current and succeeding
%                          pages to STYLE
%
%  \thispagestyle{STYLE} : sets the page style of the current page only
%                          to STYLE
%
%  To define a page style STYLE, you must define \ps@STYLE to set the page 
%  style parameters.  
%  
%  HOW A PAGE STYLE MAKES RUNNING HEADS AND FEET:
%  
% The \ps@... command defines the macros \@oddhead, \@oddfoot,
% \@evenhead, and \@evenfoot to define the running heads and feet.  
% (See output routine.)  To make headings determined by the sectioning
% commands, the page style defines the commands \chaptermark,
% \sectionmark, etc., where \chaptermark{TEXT} is called by \chapter to
% set a mark.  The \...mark commands and the \...head macros are defined
% with the help of the following macros.  (All the \...mark commands
% should be initialized to no-ops.)

\newlength{\pagewidth}
\setlength{\pagewidth}{\textwidth}
%\addtolength\pagewidth\marginparwidth
%\addtolength\pagewidth\marginparsep

\makeatletter

\def\ps@myheadings{
\def\@oddfoot{$\overline{\hfill\hbox{\it Time-Frequency Toolbox 
Tutorial : parametric representations, \today}}$}
\def\@oddhead{}
%\makebox[\textwidth][l]{\underline{\hbox to \pagewidth{\bf
%\firstmark\hfill\thepage}}}}%
\def\@evenfoot{$\overline{\hbox{
\thepage\quad F. Auger, P. Flandrin, P. Gon\c{c}alv\`es, 
O. Lemoine\hfill}}$} %
\def\@evenhead{}
%\makebox[\textwidth][r]
%{\underline{\hbox to \pagewidth{\bf\hfill\@lhead}}}}%
\def\chaptermark##1{
%\let\protect\noexpand
\mark{}\def\@lhead{##1}}%
\def\sectionmark##1{{\let\protect\noexpand\mark{\thesection 
    \hskip 1em##1}}}}

%%%%%%%%%%%%%%%%%%%%%%%%%%%%%%%%%%%%%%%%%%%%%%%%%%%%%%%%%%%%%%%%%%%%% 

\title{Time-Frequency toolbox for M\textsc{atlab}\\[0.5 cm]
User's guide and Reference guide\\
for the parametric representations}
\author{F. Auger}
\date{\today}

\begin{document}
\pagestyle{empty}
\maketitle
\chapter{A short tutorial on parametric representations}
\section{Introduction}
Parametric time-frequency representations are sometimes an interesting alternative 
to the Fourier-based time-frequency analysis tools. They have been preferred in some 
specific applications, for example in acoustics and in hydrodynamics.
These time-frequency representations are extensions of classical parametric
power spectral density estimators, which are applied on a sliding finite length
record of the signal. However, this approach requires the validity of a local stationary 
hypothesis on the signal, and therefore leads to a rtade-off between local stationarity and 
frequency resolution. For a detailed review of power spectrum estimators for
stationary signals, see X. For an easily accessible reference on their use 
for non-stationary signals, see \cite{martin95}.

All the parametric representations proposed in the toolbox
require an estimation of the ``instantaneous autocorrelation function''
$R_x[n,\tau]=\esp{x[n]\,x[n-\tau]^*}$, and compute a time-frequency representation
$P[n,\lambda)$ using different approaches~:
\begin{itemize}
\item\textbf{The periodogram}
The periodogram is nearly equivalent to the spectrogram. Its definition is
$$
P[n,\lambda)={1\over(p+1)^2}\,Z(\lambda)^H\,R_x[n]\,Z(\lambda),
$$
where $Z(\lambda)^T=\left[{1\quad e^{\jmath2\pi\lambda}\quad\ldots\quad e^{\jmath2\pi\lambda p}}\right]$
is a vector of size $p+1$, and $R_x[n]$ is the estimated autocorrelation matrix of size $p+1$.
At row $i$ and column $j$, the element of this matrix is $R_x[n,i-j]$.

\item\textbf{The Capon estimator} 
$$
P[n,\lambda)={1\over Z(\lambda)^H\,R_x[n]^{-1}\,Z(\lambda)},
$$

\item\textbf{The estimator deduced from an AR model}
\begin{eqnarray*}
P[n,\lambda)&=&{P_e[n]\over \left|{Z(\lambda)^H\,A}\right|^2},\\
\quad{\rm with}\quad
A&=&P_e[n]\,R[n]^{-1}\,\un
\quad{\rm and}\quad
P_e[n]={1\over \un^T\,R[n]^{-1}\,\un}
\end{eqnarray*}

\item\textbf{The Lagunas estimator}
$$
P[n,\lambda)={Z(\lambda)^H\,R_x[n]^{-1}\,Z(\lambda)\over 
              Z(\lambda)^H\,R_x[n]^{-2}\,Z(\lambda)},
$$

\item\textbf{The generalized Lagunas estimator}
$$
P[n,\lambda)={Z(\lambda)^H\,R_x[n]^{-2q+1}\,Z(\lambda)\over 
              Z(\lambda)^H\,R_x[n]^{-2q}\,Z(\lambda)},
$$

\end{itemize}
\section{Experimental comparison}
So as to compare these representations, we considered a synthetic signal
\chapter{Reference guide}

This section contains detailed descriptions of the M\textsc{atlab} functions 
needed for computing parametric time-frequency reppresentations. It is therefore 
a complement to the reference guide of the time-frequency toolbox. As usual,
these descriptions are also available through the online help facility.

\vspace{2 cm}

\begin{tabular}{|p{2 cm} p{10cm}|}
\hline
\multicolumn{2}{|c|}{\bf Non-Fourier based time-frequency representations}\\
\hline
 {\ty correlmx} &  autocorrelation matrix of a stationary signal\\
 {\ty parafrep} &  parametric frequency representation of a stationary signal\\
 {\ty tfrparam} &  parametric time-frequency representation \\\hline
\end{tabular}

\newpage\input correlmx.tex
\newpage\input parafrep.tex
\begin{thebibliography}{99}
\bibitem{martin95} N. Martin, J. Mars, J. Martin, C. Chorier,
``A Capon's time-octave representation application in room acoustics,''
\emph{IEEE Trans. Sig. Proc.}, Vol 43, No 8, pp 1842-1854, Aug 1995.
\bibitem{padovese96} L.R. Padovese, N. Martin, J-M. Terriez,
``Temps-fr\'equence pour l'iden-tification des caract\'eristiques dynamiques
d'uin pyl\^one de t\'el\'eph\'erique,'' \emph{Traitement du Signal}, Vol 13, No 3,
pp 209-223, 1996.
\bibitem{leprettre96} B. Leprettre, ``Reconnaissance de signaux sismiques 
d'avalanches par fusion de donn\'ees estim\'ees dans les domaines temps,
temps-fr\'equence et polarisation,'' th\`ese de doctorat de l'universit\'e 
Grenoble I, 2 octobre 1996.
\end{thebibliography}
\end{document}
